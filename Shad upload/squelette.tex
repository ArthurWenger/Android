\documentclass{article}

\usepackage{a4wide}
\usepackage[utf8]{inputenc}
\usepackage[T1]{fontenc}
\usepackage[french]{babel}
\usepackage[babel=true]{csquotes} % guillemets français
\usepackage{graphicx}
\graphicspath{{Images/}}
\usepackage{color}
\usepackage{hyperref}
\hypersetup{colorlinks,linkcolor=,urlcolor=blue}

\usepackage{amsmath}
\usepackage{amssymb}


\title{Jeu pour android: Pacman}
\author{Shad Maleck, Arthur Wenger}
\date{\today}

\begin{document}

\maketitle % pour écrire le titre


%% Le résumé:
\begin{abstract}
  Ce document constitue un rapport pour le projet de développement mobile sur Android.
\end{abstract}

\section{Introduction}
\label{section:intro} % pour faire référence à la section ailleurs (\ref{...} voir plus bas)

Quand nous avons obtenu le sujet, à savoir, de réaliser un jeu sur android utilisant les capteur, les cartes Google Map et gérant la persistance de données, nous avions eu du mal a décider du jeu. 
Arthur a tout d'abord tenté de créer un affichage de multiple balles qui se déplaceraient selon l'inclinaison du téléphone, mais, bien qu'impressionnant en terme de complexité programmatique, cela ne répondait pas vraiment à la demande de créer un jeu.
C'est pourquoi, après une rapide recherche sur internet pour un jeu simple, nous avons décidé pour ce projet de réaliser un pacman.

\section{Présentation}
\label{section:presentation}
Au lancement, le jeu propose un menu, permettant d'accéder à trois options: Lancer une nouvelle partie, Voir les scores, et quitter le jeu.

\subsection{Nouvelle partie}
\label{subsection:1_1}
Quand on lance une nouvelle partie, s'affiche à l'écran un avatar, l'éponyme Pacman.
Celui ci ce trouve au centre d'un labyrinthe, généré aléatoirement au lancement du jeu.
Selon l'inclinaison de l'appareil, le Pacman va se mettre à se déplacer.
On commence la partie avec 5 vies, on pert une vie à chaque fois que l'on touche un ennemi (fantôme), et on en gagne une pour chaque coeur que l'on absorbe.
L'objectif est d'arriver à la sortie du labyrinthe, en ayant absorbé le maximum de pièces, semées danns le labyrinthe.
Si le Pacman pert toutes ses vies, la partie est perdue, et s'affiche à l'écran le score, la position du joueur au début du jeu et son rang dans les scores, le tout sur une carte.

\subsection{Scores}
\label{subsection:1_2}
Quand on sélectionne "Voir les scores", s'affiche à l'écran une liste de scores, ordonnée par la valeur du score, selon l'ordre décroissant.
Si on clique sur l'un d'entre eux, l'appareil affiche une carte, centrée sur le marqueur, situé là où se trouvait le joueur au début de la partie, avec un marqueur pour chaque autre score.

\subsection{Quitter}
\label{subsection:1_3}
Quand on sélectionne "Quitter", on quitte l'application.

\section{Architecture de l'application}
\label{section:architecture}
Cette application a été distribuée autant que possible, afin de la rendre la plus lisible possible.
Elle est composée de cinq "grandes" activitées: MainActivity, GameActivity, GameOverActivity, ScoresActivity et MapsActivity.

\subsection{MainActivity}
\label{subsection:2_1}
MainActivity est, comme on pourrait le penser, l'activité correspodant au menu principal.
\`A sa création, la musique de fond se lance.
\begin{center}
  \includegraphics[scale=0.5]{MainActivity_onCreate.png}
\end{center}
Elle contient une série de méthode permettant de lancer d'autres activités, à savoir GameActivity ou ScoresActivity, ou de quitter.

\subsection{GameActivity}
\label{subsection:2_2}
GameActivity est l'activité correpondant au jeu lui-même.
\`A sa création, on récupère la location du joueur, on récupère les données de l'accéleromètre et on met l'application en plein écran.
\begin{center}
  \includegraphics[scale=0.5]{GameActivity_onCreate.png}
\end{center}
Elle contient des méthodes permettant de gérer les différents évennements (début de partie, fin de partie, mise en pause, reprise) et l'accéléromètre ainsi que les demandes d'authorisations pour l'accès aux capteurs.
Elle contient également des méthodes de gestion pour le menu d'options. Elle se sert des classes contenues dans le dossier game.

\subsection{GameOverActivity}
\label{subsection:2_3}
GameOverActivity est l'activité lancée à la fin d'une partie.
\`A sa création, le score réalisé dans la partie qui vient d'être perdue est ajouté à la base de données des scores, si on a réussi à localiser l'appareil au début de la partie.
\begin{center}
  \includegraphics[scale=0.5]{GameOverActivity_onCreate.png}
\end{center}
Elle contient des méthodes qui gèrent l'affichages et les deux boutons. Elle se sert fortement de la classe DBManager.

\subsection{ScoresActivity}
\label{subsection:2_4}
ScoresActivity est l'activité correspondant à l'affichage des scores.
\`A sa création on récupère la base de données contenant les scores.
\begin{center}
  \includegraphics[scale=0.5]{ScoresActivity_onCreate.png}
\end{center}
Elle ne contient qu'une seule autre méthode, qui permet d'afficher les différents scores, et de lancer MapsActivity si on sélectionne l'un d'entre eux. Elle se sert fortement de la classe DBManager.

\subsection{MapsActivityActivity}
\label{subsection:2_5}
\`A sa création, on se prépare à récupérer une carte.
\begin{center}
  \includegraphics[scale=0.5]{MapsActivity_onCreate.png}
\end{center}
Elle contient deux autres méthode permettant de gérer la carte une fois que celle-ci est disponble et de créer l'indicteur correspondant au score sélectionné pour le lancement de l'activité.

%%% La bibliographie:
%\bibliographystyle{plain}
%\bibliography{ma_biblio}

\end{document}
%D'abord, une liste à \textbf{puces}:
%\begin{itemize}
%\item la documentation de \textit{Swing}~\cite{swingDoc}
%  est bien écrite,
%\item celle de \textit{Tkinter}~\cite{tkinterDoc} aussi!
%\end{itemize}
% \cite{...} permet de faire référence à des éléments de la
% bibliographie.

%\`A présent, une liste numérotée:
%\begin{enumerate}
%\item premier item
%\item second item
%\end{enumerate}

%\subsection{Une autre sous-section}
%Pour écrire du code, on peut par exemple utiliser l'environnement
%\textit{verbatim}:
%\begin{verbatim}
%public class Main {
%   public static void main(String[] args) {
%      System.out.println("Hello World!");
%   }
%}
%\end{verbatim}

%\section{Une autre section}

% \ref{...} permet de faire référence à un élément défini
% ailleurs dans le document (voir \label{...} plus haut).
%Contrairement à la section~\ref{section:hello},
%moi je dis: \textit{Coucou!}
